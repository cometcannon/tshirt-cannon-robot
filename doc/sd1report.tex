\documentclass[letterpaper,12pt]{article}

\usepackage[margin=1in]{geometry}
\usepackage{tikz}

\usetikzlibrary{shapes.geometric, arrows, fit}

\newcommand{\AxisRotator}[1][rotate=0]{\tikz [x=0.25cm,y=0.60cm,line width=.2ex,-stealth,#1] \draw (0,0) arc (-150:150:1 and 1);}

\begin{document}
\title{EE 4388 Senior Design I\\Semester Report}
\author{Hazen Eckert \and Omar Hasan \and Ryan Marcotte \and Ridhwaan Rahman}
\maketitle
\tableofcontents
\newpage

\section{Abstract}
\noindent This project is focused on the design and implementation of a teleoperated, wheeled mobile robot that is capable of launching a t-shirt projectile. The robot will be holonomic and will consist of a rigid, rectangular chassis and a mounted cannon powered by $\textrm{CO}_2$. A human user will control all robot related tasks, including remotely driving, aiming, and launching the cannon. Both low-level motor controls and high-level coordination of motor speeds will be designed for controlled locomotion. Two on-board microcontrollers will be networked to the operator's computer, and video captured from a robot-mounted camera will be streamed to a heads-up display, which will allow for real-time control and monitoring of the robot. We will demonstrate the constructed chassis design as well as some prototypical control tasks and video streaming.

\section{Introduction}
\subsection{Subject and Purpose}
\subsection{Problem Statement and Design Objectives}
\subsection{Design Process}
\subsection{Final Results}

\section{Conceptual and Preliminary Design}
\subsection{Problem Analysis}
\subsection{Decision Analysis}

\section{Description of Solution}

\section{System Components}
\begin{figure}[h!]
  \centering
  \tikzstyle{block} = [rectangle, ultra thick, rounded corners, minimum width=2cm, minimum height=1cm,text centered, draw=black]
  \tikzstyle{dcvolt} = [solid, ultra thick, red];
  \tikzstyle{signal} = [solid, ultra thick, ->, >=stealth];

  \begin{tikzpicture}
    %\draw[help lines] (-6, -12) grid (12,1);
    \node (circuitbreaker) [block, align=center] {\textbf{Circuit}\\\textbf{Breaker}};
    \node (boost) [block, align=center, below of=circuitbreaker, xshift=-4cm, yshift=-2cm] {\textbf{Boost}\\\textbf{Converter}};
    \node (buck) [block, align=center, right of=boost, xshift=7cm] {\textbf{Buck}\\\textbf{Converter}};
    \node (mcu) [block, below of=buck, yshift=-1.25cm] {\textbf{MCU}};
    \node (triggersolenoid) [block, align=center, below of=boost, yshift=-3.5cm, xshift=-1.5cm] {\textbf{Trigger}\\\textbf{Solenoid}};
    \node (pressuresolenoid) [block, align=center, right of=triggersolenoid, xshift=2cm] {\textbf{Pressure}\\\textbf{Solenoid}};
    \node (encoders) [block, right of=pressuresolenoid, xshift=2cm] {\textbf{Encoders}};
    \node (pressuresensor) [block, align=center, right of=encoders, xshift=2cm] {\textbf{Pressure}\\\textbf{Sensor}};
    \node (camera) [block, align=center, right of=pressuresensor, xshift=2cm] {\textbf{Camera}\\\textbf{Module}};
    \node (esc) [block, right of=camera, xshift=2cm] {\textbf{ESC}};
    \node (motors) [block, below of=esc, xshift=-4.5cm, yshift=-1cm] {\textbf{Motors}};

    \draw [dcvolt] (circuitbreaker.north) -- +(0,0.5) node[anchor=south] {\textbf{+12V}};
    \draw [dcvolt] (circuitbreaker.south) -- +(0,-0.9) node[anchor=west, yshift=0.5cm] {\textbf{+12V}};
    \draw [dcvolt] (boost.north) -- +(0,1) -| +(4,1) node {};
    \draw [dcvolt] (buck.north) -- +(0,1) -| +(-4,1) node {};
    \draw [dcvolt] (esc.north) -- +(0,5.55) -| +(-5.5,5.55) node {};
    \draw [dcvolt] (buck.south) -- node[anchor=west, yshift=0.2cm] {\textbf{+5V}} (mcu.north);
    \draw [dcvolt] (encoders.north) -- +(0,2.7) -| +(3.5,2.7) node {};
    \draw [dcvolt] (pressuresensor.north) -- +(0,0.2) -| +(-3,0.2) node {};
    \draw [dcvolt] (camera.north) -- +(0,2.65) -| +(-2.5,2.65) node{};
    \draw [dcvolt] (boost.south) -- +(0,-1) node[anchor=west, yshift=0.5cm] {\textbf{+24V}};
    \draw [dcvolt] (triggersolenoid.north) -- +(0,2.4) -| +(1.5,2.4) node {};
    \draw [dcvolt] (pressuresolenoid.north) -- +(0,2.4) -| +(-1.5,2.4) node {};
    \draw [signal] (pressuresensor.east) -- +(0.5,0) -| +(0.5,0.75) -| +(-0.31, 0.75) -- (mcu.295);
    \draw [signal] (mcu.320) -- +(0,-0.8) -| +(3.5,-0.8) -| +(3.5,-1.75) -- (esc.west);
    \draw [signal] (encoders.east) -- +(0.5,0) -| +(0.5,0.95) -| +(2.39,0.95) -- (mcu.south);
    \draw [signal] (mcu.240) -- +(0,-0.6) -| +(-4.8,-0.6) -| +(-4.8,-1.75) -- (pressuresolenoid.east);
    \draw [signal] (mcu.220) -- +(0,-0.4) -| +(-7.4,-0.4) -| +(-7.4,-1.75) -- (triggersolenoid.east);
    \draw [signal] (esc.south) -- +(0,-1.5) -- (motors.east);
    \draw [signal] (motors.west) -- +(-3.5,0) -| (encoders.south);
  \end{tikzpicture}
  \caption{Electrical System Diagram}
  \label{fig:e_system}
\end{figure}

\begin{figure}[h!]
  \centering

  \tikzstyle{block} = [rectangle, ultra thick, rounded corners, minimum width=3cm, minimum height=2cm,text centered, draw=black]
\tikzstyle{container} = [draw, ultra thick, rectangle, rounded corners, dashed, text width=11em, inner sep=3.5em]
  \tikzstyle{arrow} = [thick, ->, ultra thick, >=latex]
  \begin{tikzpicture}[node distance=2cm]
    \node (input) [align=center, xshift=-1cm] {\textbf{User}\\\textbf{Inputs}};
    \node (output) [right of=input, align=center] {\textbf{Video}\\\textbf{Display}};
    \node (computer) [block, yshift=-2cm] {\textbf{Computer}};
    \node (ar9331) [block, right of=computer, xshift=9cm, yshift=-2cm, align=center] {\textbf{Atheros}\\\textbf{AR9331}};
    \node (atmega) [block, below of=ar9331, yshift=-2cm, align=center] {\textbf{Arduino}\\\textbf{ATmega32u4}};
    \node (rpi) [block, right of=computer, above of=ar9331, xshift=-2cm, yshift=2cm] {\textbf{Raspberry Pi}};

    \node (yun) [container, fit=(atmega) (ar9331)] {};
    \node (yun_text) [yshift=-0.3cm] at (yun.north) {\textbf{Yun}};

    \draw [arrow] (computer.336) -- node[anchor=east, align=center, yshift=-5mm] {\textbf{Velocity}\\\textbf{Commands}} (ar9331.188);
    \draw [arrow] (ar9331.172) -- node[anchor=west, align=center, xshift=-12mm, yshift=5mm] {\textbf{Status Updates}} (computer.352);
    \draw [arrow] (ar9331.250) -- node[anchor=east, align=center] {\textbf{GPIO}\\\textbf{Commands}} (atmega.110);
    \draw [arrow] (atmega.70) -- node[anchor=west, align=center] {\textbf{Sensor}\\\textbf{Readings}} (ar9331.290);
    \draw [arrow] (rpi.172) -- node[anchor=east, align=center, yshift=5mm] {\textbf{Video}\\\textbf{Stream}} (computer.24);
    \draw [arrow] (computer.8) -- node[anchor=west, align=center, xshift=-8mm, yshift=-5mm] {\textbf{Video}\\\textbf{Commands}} (rpi.188);
    \draw [arrow] (input) -- (computer.134);
    \draw [arrow] (computer.46) -- (output);
  \end{tikzpicture}
  \caption{System Overview}
  \label{fig:system_diagram}
\end{figure}


\section{Project Implementation and Assessment}
\section{Scope of Work}
\section{Ethics}
\noindent The intent of the t-shirt cannon robot is to aid in the distribution of t-shirt during public events like basketball games and pep rallies in a gym setting. To prevent harm, there will be several precautions that will be taken. Firstly, there will be a hard and soft master kill switch to shut down the robot in case of an emergency. Secondly, the network in which the Raspberry Pi, Arduino Yun, and the laptop are communicating will be password protected, so no one can simply hack into the system. Thirdly, the user who will be manually driving the robot wirelessly will be trained and be monitoring the vehicle in his/her line of sight. Finally, the robot will be decorated so that it will not intimidate or be invisible to the public. 

\section{Cost Estimation}
\section{Project Management Summary}
\section{Conclusion}
\noindent The Teleoperated Mobile Garment Accelerator is designed to promote and bring awareness to the engineering field. The décor and first person vision of the robot will engage the audience in public events.\newline

\noindent In Senior Design II, we will construct a mount to affix the cannon to the chassis. Additionally we will design a reloading mechanism for automatic t-shirt reloading of the cannon. Rigorous testing of the robot’s movement, firing mechanism, and network security will be performed.
\end{document}


