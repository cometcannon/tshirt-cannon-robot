\documentclass[letterpaper,12pt]{article}

\usepackage[margin=1in]{geometry}
\usepackage{tikz}

\usetikzlibrary{shapes.geometric, arrows, fit}

\newcommand{\AxisRotator}[1][rotate=0]{\tikz [x=0.25cm,y=0.60cm,line width=.2ex,-stealth,#1] \draw (0,0) arc (-150:150:1 and 1);}

\begin{document}
\title{EE 4388 Senior Design I\\Semester Report}
\author{Hazen Eckert \and Omar Hasan \and Ryan Marcotte \and Ridhwaan Rahman}
\maketitle
\tableofcontents
\newpage

\section{Abstract}
\noindent This project is focused on the design and implementation of a teleoperated, wheeled mobile robot that is capable of launching a t-shirt projectile. The robot will be holonomic and will consist of a rigid, rectangular chassis and a mounted cannon powered by $\textrm{CO}_2$. A human user will control all robot related tasks, including remotely driving, aiming, and launching the cannon. Both low-level motor controls and high-level coordination of motor speeds will be designed for controlled locomotion. Two on-board microcontrollers will be networked to the operator's computer, and video captured from a robot-mounted camera will be streamed to a heads-up display, which will allow for real-time control and monitoring of the robot. We will demonstrate the constructed chassis design as well as some prototypical control tasks and video streaming.

\section{Introduction}

\subsection{Subject and Purpose}
\noindent Our task was to design a teleoperated, wheeled mobile robot that can launch t-shirts safely at university events. The primary purpose of the robot will be to promote the Erik Jonsson School of Engineering and inspire undergraduates engineering students to gain hands-on experience with university-sponsered engineering projects.

\subsection{Design Objectives}
\noindent The design will meet the following objects:
\begin{itemize}
    \item Operation time \textgreater 10 minutes
    \item Remote Control for up to 50 feet
    \item User Control of Robot Velocities and Cannon Triggering Mechanism
    \item Fire t-shirts between 20 and 150 feet
    \item Send live video stream of the robot's First Person View back to the user
\end{itemize}

\subsection{Design Process}
\noindent We began designing our robot by determining that the drive system would be a 4-wheel omnidirectional drive with a t-shirt launching mechanism attached rigidly to the robot's chassis. Then, a t-shirt launching mechanism was chosen. An electrical system for wirelessly receiving and transmitting commands and controlling the robot's movements and t-shirt launching mechanism was designed. Finally, a software system for handling the robot-user interface was designed. 

\subsection{Final Results}
\noindent We were able to assemble the robot chassis and demonstrate video streaming using a Raspberry Pi and camera module. We also determined and purchased the necessary components for our robot to operate. Finally, we've nearly completed remote control of robot velocities. 

\section{Conceptual and Preliminary Design}

\subsection{Problem Analysis}
\noindent The robot must wirelessly receive and perform velocity and t-shirt launching commands. To perform the commands, the robot must first have a mechanism by which to control its velocity and also a triggering mechanism for the t-shirt launcher. The robot must have a wireless receiver on-board and a camera for first person view streaming. A central controller should coordinate all robot tasks on-board and a power source for both the t-shirt launching mechanism as well as the components dedicated to maneuvering the robot. Following specifications will solve the problem as thus described:
\begin{itemize}
    \item 4-wheel omnidirectional drive
    \item Pneumatic t-shirt Launcher with a shot range between 20 and 150 feet
    \item Single power source for all electrical components
    \item Single microcontroller board that includes a microcontroller and a microprocessor for embedded linux and wireless communication
    \item Highly reconfigurable IP camera
\end{itemize}

\subsection{Decision Analysis}
\noindent We decided for the robot to have 4-wheel omnidirectional drive to allow the user ease of control over the robot's pose. In this configuration, the user will be able to separately control the robot's angular velocity and translational velocity to achieve the desired pose. We chose this configuration over others because it is simpler to implement and pre-existing solutions are readily available at certain vendors. \\

\noindent The robot's original application is to be used during the halftime of basketball games to entertain the crowd. Given the size of UTD's basketball court and bleachers, a shot range between 20 and 150 feet is reasonable for this application. For convenience, a pneumatic cannon capable of firing single shots is a widely available commodity that can be retrofitted for our purposes. \\

\noindent A single power source would be ideal for powering the robot's 4 motors, microcontroller, and other electronic devices. We decided to go with a single power source rather and use up-conversion and down-conversion voltage regulators rather than using multiple power sources. \\ 

\noindent A central microcontroller is needed to control the robot's motion and triggering mechanism. Furthermore, since the robot will need to take wireless commands as input, it would be ideal to wirelessly program the robot's microcontroller as well. We chose this solution over using separate microcontrollers for high and low level commands. \\

\section{Description of Solution}
\subsection{T-Shirt Cannon}
The requirements for our cannon are that it should be electronically trigger-able, able to shoot a T-Shirt 20 to 150 feet from a 45 degree angle, and be able to make 20 shots on a single charge. The cannon we chose is CO2 based and triggers pneumatically. Other options included a mechanical throwing mechanism or building the cannon ourselves, but those options would have taken too much time away from designing the other components. A 20oz CO2 tank provides enough capacity for at least 20 shots at maximum range. In order to trigger electronically, we replaced the manual pneumatic trigger with a solenoid valve. An electronic regulator would not fit within our budget so we decided to use a pressure transducer and a solenoid valve to allow us to set the firing pressure of the cannon, and therefore the range, electronically. The pressure transducer provides the current pressure in the cannon’s reservoir and the solenoid allows us to shut off the flow of CO2 once the desired pressure is achieved.\\ 
\subsection{Chassis}
The chassis for our robot had to move holonomically, have maximum speed of 10 ft/s, be safe for gymnasium floors, and have a load capacity of approximately 60 pounds. These requirements led us to choose a four wheeled mecanum drive system mounted on an aluminum chassis. We chose mecanum wheels because they have the highest load capacity and simplest control algorithm of the holonomic drive solutions we could afford. Mecaum wheels have rollers attached at an angle around the wheel. The wheels can operate like a screw and enable lateral movement and rotating in place. This comes at the cost of a loss of efficiency and each wheel requires a separate motor, gearbox, quadrature encoder, and motor controller. \\
The motors and gearboxes that came with the chassis provide more than enough torque to achieve 10 ft/s. However, in order to eliminate drift and provide smoother controls, we decided to include quadrature encoders to provide velocity feedback for a closed loop control loop. While the motor controllers we chose can provide 40A continuous current to each motor, our power distribution system is only capable of delivering 30A to each motor.\\
\subsection{Control Electronics}
An Arduino Yun microcontroller provides the robot with a WiFi connection and a low level microcontroller. Other options included Texas Instruments based microcontrollers, but because we wanted simplicity of programming, we went with the Arduino. The Yun has both an ATmega microcontroller for low level control as well as an Atheros microprocessor chip that runs the Linux based OpenWrt OS that provides networking capabilities. The Atheros runs our application that receives velocity, triggering, and other commands over WiFi and processes them into low level commands for the ATmega. The ATmega handles hardware interrupt, monitors electrical components, and executes commands given by the Atheros device. Connected to the ATmega are the motor controllers, which are controlled via PWM, the quadrature encoder, which sense rotations of the wheels for velocity feedback, and the pneumatic circuits, which provide electronic triggering and range adjustment. \\
In the software of the control electronics are several safety power shutoffs. If communication ceases between the Atheros and the ATmega or between the user and the Atheros, the robot with shut off the motors. This is achieved through a “heartbeat“ protocol, which resets a timer every time a message is received. If no message is received before the timer expires, the connection is assumed lost and power is shut off.\\
\subsection{User Interface}
The user interface runs on a personal computer running any operating system with a Java virtual machine. An X-Box controller provides the user with a familiar and intuitive control scheme. The graphical user interface running on the personal computer updates the user with diagnostic information such as current firing pressure, battery charge, and connection strength. This interface also presents the user with a video stream from the robot’s point of view. A Linux based single board computer, the Raspberry Pi, is streaming this video from a webcam on the robot.\\
\subsection{Power Distribution System}
The electronic devices on the robot require three different voltages: the pneumatic components must have 24VDC, the motors need 12VDC, and the control electronics operate at 5VDC. Since the majority of the current would be drawn at 12V we chose a 12VDC lead acid battery and used a buck converter to obtain 5VDC and a boost converter to achieve 24VDC. A common automotive fuse box is used to distribute the 12VDC power among up to 12 circuits at up to 30A. The battery is able to source up to 300A at any time and can source 60A for 10 minutes straight before discharging. Since our motors will draw at least 2A at free current, we expect that once the robot achieves a constant velocity, each motor will draw around 10A. Thus, we expect our battery to give us enough power to allow the robot to operate for at least 10 minutes at a time.\\
In the interest of safety, we have included in the power distribution system a kill switch that is triggered by physically pulling a bright orange lanyard. This cuts the power from the battery to all systems of the robot.\\

\section{System Components}
\begin{figure}[h!]
  \centering
  \tikzstyle{block} = [rectangle, ultra thick, rounded corners, minimum width=2cm, minimum height=1cm,text centered, draw=black]
  \tikzstyle{dcvolt} = [solid, ultra thick, red];
  \tikzstyle{signal} = [solid, ultra thick, ->, >=stealth];

  \begin{tikzpicture}
    %\draw[help lines] (-6, -12) grid (12,1);
    \node (circuitbreaker) [block, align=center] {\textbf{Circuit}\\\textbf{Breaker}};
    \node (boost) [block, align=center, below of=circuitbreaker, xshift=-4cm, yshift=-2cm] {\textbf{Boost}\\\textbf{Converter}};
    \node (buck) [block, align=center, right of=boost, xshift=7cm] {\textbf{Buck}\\\textbf{Converter}};
    \node (mcu) [block, below of=buck, yshift=-1.25cm] {\textbf{MCU}};
    \node (triggersolenoid) [block, align=center, below of=boost, yshift=-3.5cm, xshift=-1.5cm] {\textbf{Trigger}\\\textbf{Solenoid}};
    \node (pressuresolenoid) [block, align=center, right of=triggersolenoid, xshift=2cm] {\textbf{Pressure}\\\textbf{Solenoid}};
    \node (encoders) [block, right of=pressuresolenoid, xshift=2cm] {\textbf{Encoders}};
    \node (pressuresensor) [block, align=center, right of=encoders, xshift=2cm] {\textbf{Pressure}\\\textbf{Sensor}};
    \node (camera) [block, align=center, right of=pressuresensor, xshift=2cm] {\textbf{Camera}\\\textbf{Module}};
    \node (esc) [block, right of=camera, xshift=2cm] {\textbf{ESC}};
    \node (motors) [block, below of=esc, xshift=-4.5cm, yshift=-1cm] {\textbf{Motors}};

    \draw [dcvolt] (circuitbreaker.north) -- +(0,0.5) node[anchor=south] {\textbf{+12V}};
    \draw [dcvolt] (circuitbreaker.south) -- +(0,-0.9) node[anchor=west, yshift=0.5cm] {\textbf{+12V}};
    \draw [dcvolt] (boost.north) -- +(0,1) -| +(4,1) node {};
    \draw [dcvolt] (buck.north) -- +(0,1) -| +(-4,1) node {};
    \draw [dcvolt] (esc.north) -- +(0,5.55) -| +(-5.5,5.55) node {};
    \draw [dcvolt] (buck.south) -- node[anchor=west, yshift=0.2cm] {\textbf{+5V}} (mcu.north);
    \draw [dcvolt] (encoders.north) -- +(0,2.7) -| +(3.5,2.7) node {};
    \draw [dcvolt] (pressuresensor.north) -- +(0,0.2) -| +(-3,0.2) node {};
    \draw [dcvolt] (camera.north) -- +(0,2.65) -| +(-2.5,2.65) node{};
    \draw [dcvolt] (boost.south) -- +(0,-1) node[anchor=west, yshift=0.5cm] {\textbf{+24V}};
    \draw [dcvolt] (triggersolenoid.north) -- +(0,2.4) -| +(1.5,2.4) node {};
    \draw [dcvolt] (pressuresolenoid.north) -- +(0,2.4) -| +(-1.5,2.4) node {};
    \draw [signal] (pressuresensor.east) -- +(0.5,0) -| +(0.5,0.75) -| +(-0.31, 0.75) -- (mcu.295);
    \draw [signal] (mcu.320) -- +(0,-0.8) -| +(3.5,-0.8) -| +(3.5,-1.75) -- (esc.west);
    \draw [signal] (encoders.east) -- +(0.5,0) -| +(0.5,0.95) -| +(2.39,0.95) -- (mcu.south);
    \draw [signal] (mcu.240) -- +(0,-0.6) -| +(-4.8,-0.6) -| +(-4.8,-1.75) -- (pressuresolenoid.east);
    \draw [signal] (mcu.220) -- +(0,-0.4) -| +(-7.4,-0.4) -| +(-7.4,-1.75) -- (triggersolenoid.east);
    \draw [signal] (esc.south) -- +(0,-1.5) -- (motors.east);
    \draw [signal] (motors.west) -- +(-3.5,0) -| (encoders.south);
  \end{tikzpicture}
  \caption{Electrical System Diagram}
  \label{fig:e_system}
\end{figure}

\begin{figure}[h!]
  \centering

  \tikzstyle{block} = [rectangle, ultra thick, rounded corners, minimum width=3cm, minimum height=2cm,text centered, draw=black]
\tikzstyle{container} = [draw, ultra thick, rectangle, rounded corners, dashed, text width=11em, inner sep=3.5em]
  \tikzstyle{arrow} = [thick, ->, ultra thick, >=latex]
  \begin{tikzpicture}[node distance=2cm]
    \node (input) [align=center, xshift=-1cm] {\textbf{User}\\\textbf{Inputs}};
    \node (output) [right of=input, align=center] {\textbf{Video}\\\textbf{Display}};
    \node (computer) [block, yshift=-2cm] {\textbf{Computer}};
    \node (ar9331) [block, right of=computer, xshift=9cm, yshift=-2cm, align=center] {\textbf{Atheros}\\\textbf{AR9331}};
    \node (atmega) [block, below of=ar9331, yshift=-2cm, align=center] {\textbf{Arduino}\\\textbf{ATmega32u4}};
    \node (rpi) [block, right of=computer, above of=ar9331, xshift=-2cm, yshift=2cm] {\textbf{Raspberry Pi}};

    \node (yun) [container, fit=(atmega) (ar9331)] {};
    \node (yun_text) [yshift=-0.3cm] at (yun.north) {\textbf{Yun}};

    \draw [arrow] (computer.336) -- node[anchor=east, align=center, yshift=-5mm] {\textbf{Velocity}\\\textbf{Commands}} (ar9331.188);
    \draw [arrow] (ar9331.172) -- node[anchor=west, align=center, xshift=-12mm, yshift=5mm] {\textbf{Status Updates}} (computer.352);
    \draw [arrow] (ar9331.250) -- node[anchor=east, align=center] {\textbf{GPIO}\\\textbf{Commands}} (atmega.110);
    \draw [arrow] (atmega.70) -- node[anchor=west, align=center] {\textbf{Sensor}\\\textbf{Readings}} (ar9331.290);
    \draw [arrow] (rpi.172) -- node[anchor=east, align=center, yshift=5mm] {\textbf{Video}\\\textbf{Stream}} (computer.24);
    \draw [arrow] (computer.8) -- node[anchor=west, align=center, xshift=-8mm, yshift=-5mm] {\textbf{Video}\\\textbf{Commands}} (rpi.188);
    \draw [arrow] (input) -- (computer.134);
    \draw [arrow] (computer.46) -- (output);
  \end{tikzpicture}
  \caption{System Overview}
  \label{fig:system_diagram}
\end{figure}


\section{Project Implementation and Assessment}
\section{Scope of Work}
\section{Ethics}
\noindent The intent of the t-shirt cannon robot is to aid in the distribution of t-shirt during public events like basketball games and pep rallies in a gym setting. To prevent harm, there will be several precautions that will be taken. Firstly, there will be a hard and soft master kill switch to shut down the robot in case of an emergency. Secondly, the network in which the Raspberry Pi, Arduino Yun, and the laptop are communicating will be password protected, so no one can simply hack into the system. Thirdly, the user who will be manually driving the robot wirelessly will be trained and be monitoring the vehicle in his/her line of sight. Finally, the robot will be decorated so that it will not intimidate or be invisible to the public. 

\section{Cost Estimation}
\section{Project Management Summary}
\section{Conclusion}
\noindent The Teleoperated Mobile Garment Accelerator is designed to promote and bring awareness to the engineering field. The décor and first person vision of the robot will engage the audience in public events.\newline

\noindent In Senior Design II, we will construct a mount to affix the cannon to the chassis. Additionally we will design a reloading mechanism for automatic t-shirt reloading of the cannon. Rigorous testing of the robot’s movement, firing mechanism, and network security will be performed.
\end{document}


