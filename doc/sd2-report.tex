\documentclass[letterpaper,12pt]{article}

\usepackage[margin=1.0in]{geometry}
\usepackage{multirow}
\usepackage{tikz}
\usepackage{amsmath}
\usepackage{hyperref}
\usepackage{pdfpages}
\usepackage{standalone}
\usepackage{fancyhdr}

\pagestyle{fancy}
\fancyhf{}
\fancyhead[RE,LO]{EE 4389 Senior Design II Final Report}
\fancyhead[LE,RO]{Teleoperated Mobile Garment Accelerator}
\fancyfoot[RE,LO]{Eckert, Hasan, Marcotte, and Rahman}
\fancyfoot[LE,RO]{\thepage}

\usetikzlibrary{shapes.geometric, arrows, fit}

\setlength\parindent{0pt}

\newcommand{\specialcell}[2][c]{\begin{tabular}[#1]{@{}c@{}}#2\end{tabular}}
\newcommand{\xxx}[1]{{\color{red}\bf #1}}
\newcommand{\AxisRotator}[1][rotate=0]{\tikz [x=0.25cm,y=0.60cm,line width=.2ex,-stealth,#1] \draw (0,0) arc (-150:150:1 and 1);}

\begin{document}

\title{\textbf{EE 4389 Senior Design II\\Semester Report}}
\author{Hazen Eckert \hspace{3mm} Omar Hasan \hspace{3mm} Ryan Marcotte \hspace{3mm} Ridhwaan Rahman}
\date{11 May 2015}
\maketitle

\begin{center}
    \includegraphics[width=15cm]{./pics/chassis/robot.jpg}
\end{center}

\pagebreak
\tableofcontents
\pagebreak

\section{Introduction}
\label{sec:intro}

\subsection{Purpose}
\label{sec:purpose}
Our task was to design a teleoperated, wheeled mobile robot that can launch
t-shirts safely at university events. The primary purpose of the robot will be
to promote the Erik Jonsson School of Engineering and inspire undergraduate
engineering students to gain hands-on experience with university-sponsored
engineering projects.

\subsection{Problem Statement and Design Objectives}
\label{sec:probstatement}

The design will meet the following objectives:
\begin{itemize}
    \item Operation time \textgreater 10 minutes
    \item Remote control for up to 50 feet
    \item User control of robot velocities and cannon triggering mechanism
    \item Fire t-shirts between 20 and 150 feet
    \item Send live video stream of the robot's First Person View back to the user
\end{itemize}

\subsection{Design Process Summary}
\label{sec:designprocesssummary}
We began designing our robot by determining that the drive system would be
a 4-wheel omnidirectional drive with a t-shirt launching mechanism attached
rigidly to the robot's chassis. Then, a t-shirt launching mechanism was chosen.
An electrical system for wirelessly receiving and transmitting commands and
controlling the robot's movements and t-shirt launching mechanism was designed.
Finally, a software system for handling the robot-user interface was designed.

\subsection{Final Results Summary}
\label{sec:resultssummary}
Our robot meets the specifications that were put forth in the earlier
subsections. It has been tested already on numerous occasions and has performed
the desired tasks on those occasions.

\section{Review of Conceptual and Preliminary Design}
\label{sec:conceptualpreliminarydesign}

\subsection{Problem Analysis}
\label{sec:probanalysis}
The robot must wirelessly receive and perform velocity and t-shirt launching
commands. To perform the commands, the robot must first have a mechanism by
which to control its velocity and also a triggering mechanism for the t-shirt
launcher. The robot must have a wireless receiver on-board and a camera for
first person view streaming. A central controller should coordinate all robot
tasks on-board and a power source for both the t-shirt launching mechanism as
well as the components dedicated to maneuvering the robot. The following
specifications will solve the problem as thus described:

\begin{itemize}
    \item 4-wheel omnidirectional drive
    \item Pneumatic t-shirt launcher with a shot range between 20 and 150 feet
    \item Single power source for all electrical components
    \item Single microcontroller board that includes a microcontroller and
        a microprocessor for embedded linux and wireless communication
    \item Highly reconfigurable IP camera
\end{itemize}

\subsection{Decision Analysis}
\label{sec:decisionanalysis}
We decided for the robot to have 4-wheel omnidirectional drive to allow the
user ease of control over the robot's pose. In this configuration, the user
will be able to separately control the robot's angular velocity and
translational velocity to achieve the desired pose. We chose this configuration
over others because it is simpler to implement and pre-existing solutions are
readily available at certain vendors.\\

The robot's original application is to be used during the halftime of
basketball games to entertain the crowd. Given the size of UTD's basketball
court and bleachers, a shot range between 20 and 150 feet is reasonable for
this application. For convenience, a pneumatic cannon capable of firing single
shots is a widely available commodity that can be retrofitted for our purposes.\\

A single power source would be ideal for powering the robot's 4 motors,
microcontroller, and other electronic devices. We decided to go with a single
power source and use up-conversion and down-conversion voltage regulators
rather than using multiple power sources.\\

A central microcontroller is needed to control the robot's motion and
triggering mechanism. Furthermore, since the robot will need to take wireless
commands as input, it would be ideal to wirelessly program the robot's
microcontroller as well. We chose this solution over using separate
microcontrollers for high and low level commands.\\

\section{Basic Solution Description}
\label{sec:basicsoldesc}

\xxx{Schematics and flow sheets defining all system components and their relationships}\\
\xxx{Analysis/balance of key parameters (voltage, current, power, timing or computing requirements, mechanical parameters, etc.)}\\
\xxx{Component sizing and component-level specifications}\\
\xxx{Initial system performance estimates}

\subsection{T-Shirt Cannon}
The requirements for our cannon are that it should be electronically
trigger-able, able to shoot a t-shirt 20 to 150 feet from a 45 degree angle,
and be able to make 20 shots on a single charge. The cannon we chose is CO$_2$
based and triggers pneumatically. Other options included a mechanical throwing
mechanism or building the cannon ourselves, but those options would have taken
too much time away from designing the other components. A 20oz CO$_2$ tank
provides enough capacity for at least 20 shots at maximum range. In order to
trigger electronically, we replaced the manual pneumatic trigger with
a solenoid valve. An electronic regulator would not fit within our budget so we
decided to use a pressure transducer and a solenoid valve to allow us to set
the firing pressure of the cannon, and therefore the range, electronically. The
pressure transducer provides the current pressure in the cannon's reservoir and
the solenoid allows us to shut off the flow of CO$_2$ once the desired pressure
is achieved.\\

\subsection{Chassis}
The chassis for our robot has to move holonomically, have maximum speed of 10
ft/s, be safe for gymnasium floors, and have a load capacity of approximately
60 pounds. These requirements led us to choose a four wheeled mecanum drive
system mounted on an aluminum chassis. We chose mecanum wheels because they
have the highest load capacity and simplest control algorithm of the holonomic
drive solutions we could afford. Mecanum wheels have rollers attached at an
angle around the wheel. The wheels can operate like a screw and enable lateral
movement and rotating in place. This comes at the cost of a loss of efficiency
and each wheel requires a separate motor, gearbox, quadrature encoder, and
motor controller.\\

The motors and gearboxes that came with the chassis provide more than enough
torque to achieve 10 ft/s. However, in order to eliminate drift and provide
smoother controls, we decided to include quadrature encoders to provide
velocity feedback for a closed loop control loop. While the motor controllers
we chose can provide 40A continuous current to each motor, our power
distribution system is only capable of delivering 30A to each motor.\\

\subsection{Control Electronics}
An Arduino Yun microcontroller provides the robot with a WiFi
connection and a low level microcontroller. Other options included Texas
Instruments based microcontrollers, but because we wanted simplicity of
programming, we went with the Arduino. The Yun has both an ATmega
microcontroller for low level control as well as an Atheros microprocessor chip
that runs OpenWrt Linux that provides networking capabilities. The Atheros runs
our application that receives velocity, triggering, and other commands over
WiFi and processes them into low level commands for the ATmega. The ATmega
handles hardware interrupt, monitors electrical components, and executes
commands given by the Atheros device. Connected to the ATmega are the motor
controllers, which are controlled via PWM, the quadrature encoder, which sense
rotations of the wheels for velocity feedback, and the pneumatic circuits,
which provide electronic triggering and range adjustment.\\

In the software of the control electronics are several safety power shutoffs.
If communication ceases between the Atheros and the ATmega or between the user
and the Atheros, the robot with shut off the motors. This is achieved through
a “heartbeat“ protocol, which resets a timer every time a message is received.
If no message is received before the timer expires, the connection is assumed
lost and power is shut off.\\

\subsection{User Interface}
The user interface runs on a personal computer running any operating system
with a Java Virtual Machine. An X-Box controller provides the user with
a familiar and intuitive control scheme. The graphical user interface running
on the personal computer updates the user with diagnostic information such as
current firing pressure, battery charge, and connection strength. This
interface also presents the user with a video stream from the robot’s point of
view. A Linux based single board computer, the Raspberry Pi, is streaming this
video from a webcam on the robot.\\

\subsection{Power Distribution System}
The electronic devices on the robot require three different voltages: the
pneumatic components must have 24VDC, the motors need 12VDC, and the control
electronics operate at 5VDC. Since the majority of the current would be drawn
at 12V we chose a 12VDC lead acid battery and used a buck converter to obtain
5VDC and a boost converter to achieve 24VDC. A common automotive fuse box is
used to distribute the 12VDC power among up to 12 circuits at up to 30A. The
battery is able to source up to 300A at any time and can source 60A for 10
minutes straight before discharging. Since our motors will draw at least 2A at
free current, we expect that once the robot achieves a constant velocity, each
motor will draw around 10A. Thus, we expect our battery to give us enough power
to allow the robot to operate for at least 10 minutes at a time.\\

\xxx{In the interest of safety, we have included in the power distribution
system a kill switch that is triggered by physically pulling a bright orange
lanyard.  This cuts the power from the battery to all systems of the robot.}\\

\begin{figure}[h!]
  \centering
  \includestandalone[width=0.8\textwidth]{./tikz-figures/software-overview}
  \caption{Software System Diagram}
  \label{fig:system_diagram}
\end{figure}

\begin{figure}[h!]
  \centering
  \includestandalone[width=0.8\textwidth]{./tikz-figures/electroncs-overview}
  \caption{Electrical System Diagram}
  \label{fig:e_system}
\end{figure}

\section{Performance Optimization and Design of System Components}
\label{sec:optimization}
\xxx{Description of components and their component-level specifications} \\
\xxx{Design criteria used} \\
\xxx{Discussion of the technical approach used} \\
\xxx{Discussion of design details} \\
\xxx{Presentation and discussion of engineering drawings and schematics} \\
\xxx{Fabrication, construction, or production instructions and specifications} \\
\xxx{This section should provide any and all information necessary to ``build'' the component of the design you have focused on, all the way down to the number of nuts, bolts, transistors, wiring harness pin-outs, etc.} \\
\xxx{Summary of the final design results} \\
\xxx{Performance evaluation}

In order to command a desired robot velocity,
\begin{math}
  v=
  \begin{bmatrix}
    v_x \\
    v_y \\
    \omega_z
  \end{bmatrix}
\end{math}
, we calculate the wheel velocities as shown in Equations \ref{eq:rw_to_v}-\ref{eq:v_wheel_4}, according to Figure \ref{fig:robot_top_view}.

\begin{figure}[h!]
  \centering
  \includestandalone[width=0.5\textwidth]{./tikz-figures/kinematics}
  \caption{Top View of Robot}
  \label{fig:robot_top_view}
\end{figure}

\begin{equation}
  v_{wheel}=r_{wheel}\omega_{wheel}
  \label{eq:rw_to_v}
\end{equation}
\begin{equation}
  v_{wheel\,0}=v_x-v_y-(l_1+l_2)\omega_z
  \label{eq:v_wheel_1}
\end{equation}
\begin{equation}
  v_{wheel\,1}=v_x+v_y+(l_1+l_2)\omega_z
  \label{eq:v_wheel_2}
\end{equation}
\begin{equation}
  v_{wheel\,2}=v_x-v_y+(l_1+l_2)\omega_z
  \label{eq:v_wheel_3}
\end{equation}
\begin{equation}
  v_{wheel\,3}=v_x+v_y-(l_1+l_2)\omega_z
  \label{eq:v_wheel_4}
\end{equation}

\begin{table}[h!]
  \centering
  \begin{tabular}{| c | c |}
    \hline
    \textbf{Motor} & \textbf{Position} \\
    \hline
    0 & Forward Left \\
    \hline
    1 & Forward Right \\
    \hline
    2 & Rear Right \\
    \hline
    3 & Rear Left \\
    \hline
  \end{tabular}
  \caption{Motor Numbers}
  \label{tab:motor_nums}
\end{table}

\section{Project Implementation, Operation, and Assessment}
\label{sec:implopassess}

\subsection{Proposed Deliverables}
At the beginning of the semester, we outlined the following deliverables for our
project.

\begin{itemize}
\item The assembled robot, with the following functionality:
  \begin{itemize}
  \item Capability to move laterally in response to remotely issued input commands.
  \item Fixed-position cannon capable of being triggered remotely.
  \item Emergency kill switch mechanism to manually disconnect power to the
    robot's motors.
  \end{itemize}
\item The accompanying software package, which will include:
  \begin{itemize}
  \item Graphical user interface with embedded video stream from robot and
    ability to issue remote commands.
  \item Network setup to allow for communication between the user's computer and
    the robot.
  \item Utilities for user to monitor low-level robot activities.
  \end{itemize}
\end{itemize}

\subsection{Results}
We have achieved all of the deliverables outlined in the previous section. The
following sections show how we have achieved them.

\subsubsection{Robot Motion}
The robot is certainly capable of being driven remotely. It is quite responsive
to the user's controls and moves very nimbly. It has a maximum speed of
approximately 6 mph in the forward direction and 4 mph laterally. Teleoperation
of the robot's motion is one of the system's most robust components at this
point.

\subsubsection{Cannon}
The cannon is currently fixed to the frame of the robot at a 45-degree
angle. Through the use of solenoids, we are able to trigger the cannon
electronically. The user can do this by holding down both triggers on the
controller. The cannon is capable of firing up to 150 feet at full
pressure. This range can also be scaled back by decreasing the pressure in the
tank.

\subsubsection{Kill Switch}
Our kill switch mechanism is still very similar to our original implementation:
the robot's motors can be shut off by pressing a button on the circuit
breaker. Though this may not be the easiest kill switch to utilize, it is
effective and has worked well in our testing. In addition to the hardware kill
switch, we also have the capability of sending software commands to turn off the
motors. Furthermore, we have mechanisms in place to shut off the robot if it
loses its connection with the user's laptop.

\subsubsection{Graphical User Interface}
Our desktop application does have a graphical user interface that displays a
video stream and some information to the user. Though the interface is minimal,
it still has useful information and capabilities. The GUI is shown in Figure
\ref{fig:gui}.

\begin{figure}[h!]
  \centering
  \includegraphics[width=0.5\textwidth]{pics/gui.png}
  \caption{Graphical User Interface}
  \label{fig:gui}
\end{figure}

\subsubsection{Network}
The mobile ad hoc network that is set up by the microcontrollers does allow them
to communicate with the user. The network has a range of approximately 75 feet
while streaming video and approximately 150 feet without video streaming.

\subsubsection{Monitoring Utilities}
We have some minimalist utilities for monitoring the robot remotely. If the user
remotely logs into the robot, they can access information about the commands it
is receiving and executing as well as feedback about the pressure in the
cannon's tank.


\section{Final Scope of Work Statement}
\label{sec:finalscope}

\subsection{Summary of Project Work}
\subsubsection{Mechanical}
Our project had a substantial mechanical engineering component to it. We
researched, selected, and assembled the chassis and wheels. We purchased a
t-shirt cannon and modified it to be able to trigger electronically. We then
mounted the cannon to the frame of the robot.

\subsubsection{Electrical}
We designed and implemented several circuits as a part of this project. We have
electronic speed controllers that are signaled by the microcontroller and drive
the wheel motors. We built circuits that allow for electronically triggering the
cannon through the use of solenoids. We also added a piezo buzzer that acts as a
horn for the robot.

\subsubsection{Software}
We developed software for both chips on the Arduino Yun as well as on the user's
laptop. On the ATmega chip, we wrote software that receives commands from the
Atheros chip and controls the various circuits of the robot, including the speed
controllers, cannon pneumatics, and horn. We developed a daemon for the Atheros
chip that receives commands from the user via a wireless network and passes them
along to the ATmega chip. Finally, on the computer we wrote a desktop
application with a graphical user interface that can receive inputs from an XBox
controller and send them via a wireless network to the robot.

\subsection{Future Work}
\subsubsection{Network Improvements}
Under our current implementation, the user connects a laptop to a mobile ad hoc
network to which the Raspberry Pi and Arduino Yun are already connected. While
our configuration does work, its performance is suboptimal and could be
improved. The largest problem plaguing the network is that video streaming is
very bandwidth-intensive. This causes the video stream to have large amounts of
latency at times, and sometimes this latency seeks into the controls of the
robot as well. Video streaming will always be bandwidth-intensive, and video
streaming through a mobile ad hoc network will always be troublesome. One
improvement that could be made, though, is to switch the transport protocol of
the video stream from TCP to UDP. TCP has more overhead than UDP, which can
cause large amounts of latency, particularly as a link deteriorates. Even if the
video streaming performance is not optimal, changes could be made so that the
video streaming does not affect the robot controls as drastically. Namely, we
could introduce Quality of Service considerations into the network devices such
that the control packets are prioritized over the video stream packets.

\subsubsection{Automated Cannon Reloading}
One of the most desirable additions to the robot would be a mechanism that
allows automated reloading of the cannon. When we demonstrate the cannon to the
public, this is one of the most requested features. It would be very cool to
have the robot be able to fire a volley of t-shirts before having to return to
the user for reloading. Though this is highly desirable, it is also very
difficult. We discussed a few possibilities for this, including a conveyor belt
system and a Howitzer-style rotating barrels setup. Both of these would likely
require a team of mechanical engineers to implement effectively.

\subsubsection{Tilting Cannon Mount}
Another further refinement regarding the cannon is a mechanism to be able to
tilt the cannon. Currently, the cannon is fixed at a 45-degree angle. This does
work sufficiently well for our application, since we can modulate the range of
the shot through changes in the pressure in the cannon's tank. However, it would
be cool for the user to be able to tilt the cannon up and down on command. This
could be accomplished by hinging the back of the robot and using a linear
actuator to tilt the front up and down.

\subsubsection{Encoders}
We purchased encoders for each of the wheels to report estimated wheel
velocities. We experimented with them but had mixed results, largely due to
communication problems between the chips of the Arduino Yun. Fixing these issues
and installing the encoders onto the robot would allow for monitoring of the
robot speed and for closed-loop control of that speed.

\subsubsection{Remote Teloperation}
Though this falls outside the bounds of the intended application for this robot,
remote teleoperation would be a very interesting application. The user could
utilize the video stream coming from the robot to monitor the robot's progress
and send commands with the controller. We attempted to do this by connecting the
robot to the school's network. However, due to the restrictions of the network's
firewall, it is not possible to connect devices together on that particular
network. A more robust solution would be to set up an internet-connected server
dedicated to this purpose. Both the robot and the user would then just connect
to the internet and access the server. The server would be able to relay
commands from the user to the robot and send the video stream back to the user.

\subsubsection{Computer Vision}
Our original motivation for putting the camera on the robot was to provide a
sensor for some computer vision applications. We ended up not really pursuing
this due to time constraints, but there are many interesting possibilities. One
option would be to implement a ``targeting'' mechanism by which the robot could
autonomously aim and fire at targets based on their color. Another option would
be to implement visual servoing as a closed-loop control mechanism for the
robot.

\subsubsection{Cannon Failsafe}
One of our primary concerns throughout the project was ensuring the safety of
the user and the public. Because we are firing projectiles at a high rate of
speed, there is the potential for injury. The most concerning situation would be
if a person was very close to the cannon and it was triggered. Currently, we are
relying mainly on the judgment of the operator to avoid such incidents. However,
safety measures should probably be implemented to make the cannon more failsafe
in this regard. Specifically, we should have a mechanism that disables the
cannon from firing if any object is within a certain range of it. The best idea
that we had for implementing this was with a laser rangefinder that reported the
distance to the closest object. If the sensor indicates that an object is in the
path of the cannon, the MCU disables the cannon's triggering mechanism.

\subsubsection{Recharging Circuit}
\xxx{Omar, write me}

\subsubsection{Turbo Mode}
The robot has a top speed of around 6 mph in the forward direction and 4 mph
laterally. However, we have limited the speed of the robot in software so that
it can only move at approximately half of its maximum speed. We did this for
several reasons, including safety and usability. However, it would be good for
the full potential of this robot to be accessible to the user if desired. This
could be implemented in the form of a ``Turbo Mode'' in which the user holds
down a button on the controller to get a speed boost. In addition to software
changes that must occur for this to be possible, we also need to add larger
charging capacitors in order to counteract the large current draws that are
demanded during quick accelerations.

\subsubsection{Mobile Application}
As we have driven the robot around campus, we have realized that it is actually
quite burdensome to only be able to control it from a desktop application. From
a usability perspective, it would be much nicer to have the option of using a
mobile application, particularly if the user is also mobile. To avoid the
trouble of creating separate applications to support the various mobile
platforms (iOS, Android, Windows), it may be preferable to create a
browser-based application.

\section{Cost Estimation}

\label{sec:cost}
1x Arduino Yun MCU: \$65.75 \\
1x Mecanum chassis: \$1012 \\
1x T-shirt cannon: \$365 \\
2x Lead Acid Battery: \$84 \\
1x Electronic Speed Controller: \$45 \\
1x Battery Fuse: \$50 \\
1x Power Converter: \$15 \\
1x Assorted 30A Fuses: \$5.99 \\
1x Power Distribution Board: \$38.69 \\
1x Pi Camera Module: \$26.72 \\
4x Churro Tubes: \$20 \\
4x L Brackets: \$24 \\
1x Raspberry Pi Clear Camera Case: \$8.99 \\
2x CO2 Tank: \$41.98 \\
1x 2-Way 2-pos 1/4in solenoid valve: \$56 \\
1x 150W Boost Converter: \$6.10 \\
1x 200psi Pressure Transducer: \$26 \\
25x 1n4001 diode: \$3.70 \\
10x MOSFET: \$9.14 \\

4x Victor SP Speed Controller: \$240 \\
1x AQUATEK CO2 Paintball Tank CGA 320 Adapter: \$16.99 \\
1x 1/4-20 Nylock Nut, 100 (am-1160): \$4 \\
1x 1/4-20 x 1-3/4" SHCS, 50 (am-1206): \$14 \\
2x 72" C-Channel (am-3052): \$56 \\
1x 30" C-Channel (am-2222): \$18 \\
8x Vertical Corner Link for C-Channel (am-0935): \$88 \\
1x Hitec 59411 Servo Wire 50' 3 Color \$13.41 \\
1x Easy More 15Pcs 40 Pin 2.54 mm Single Row(L 11MM) Male Header \$8.60 \\
1x Sparkfun Ribbon Cable - 10 wire (15ft) \$9.95 \\
1x Goodyear EP 46508 3/8-Inch by 3-Feet 250 PSI Lead-In Rubber Air Hose with 1/4-Inch MNPT Ends \$9.06 \\
1x Raspberry Pi 2 Model B Project Board	\$44.79 \\
1x Premium Clear Case for Raspberry Pi 2 Model B Quad Core and Raspberry Pi Model B+ \$9.49 \\
1x 8GB Sandisk MicroSDHC Memory Card with SD Adapter	\$6.49 \\
1x Wireless Xbox 360 Gaming Controller \$33.49 \\
1x Charge kit \$13.99 \\
1x TP-Link TL-WN722N Wireless Adapter \$14.99 \\

Miscellaneous (Wire, nuts, bolts, etc): \$95.37 \\

Total Cost: \$2600.68

\pagebreak

\section{Project Management Summary}
\label{sec:projman}

\subsection{Gantt Chart}
\includepdf[pages={-}]{./pics/s1.pdf}
\includepdf[pages={-}]{./pics/s2.pdf}

\subsection{Work Breakdown Structure}
\includepdf[pages={-}]{./pics/wbs.pdf}

\subsection{Tasks}
Completed:
\begin{itemize}
\item Pressure sensing and control of the cannon's air tank
\item Software development on the Arduino Yun
\item Chassis Assembly
\item Video streaming
\item Solenoid as triggering mechanism
\end{itemize}

Future Work:
\begin{itemize}
    \item Add reloading capabilities. Currently the cannon has to be reloaded
        manually after every shot.
    \item Modulate angle of elevation. This can be solved by implementing
        a linear actuator.
    \item Decorate the exterior. Aesthetics of the robot are being designed by
        students in ATEC.
\end{itemize}

\subsection{Time}
Two Semesters:
\begin{itemize}
    \item Weekly meetings with Dr. Gans
    \item Weekly meetings among group to organize and assign tasks
\end{itemize}

\subsection{Budget}

Our total available budget is \$3,000. Thus far we have spent \$2600.44 on
a microcontrollers, chassis, t-shirt cannon, power distribution essentials, and
pneumatics.

\subsection{Facilities}
We conducted our work in the UTDesign Lab in SPN.

\section{Conclusion}
\label{sec:conclusion}

Our sponsor and customer, the UTDesign Program, tasked us with developing
a mobile robot capable of firing soft projectiles - such as t-shirts or
confetti. This robot is to be used at sporting events, recruitment visits, and
perhaps even graduation as a way of promoting UT Dallas, the Erik Jonsson
School, and the engineering discipline in general.\\

The robot's mecanum wheels allow it to move in any direction instantaneously.
The cannon is powered by compressed CO2. We are able to remotely control the
tank's pressure and fire the cannon through the use of solenoids. The robot has
two microcontrollers on board that are networked together. An Arduino Yun
accepts and carries out commands from the user while sending a continuous
stream of diagnostic information back to the user. A Raspberry Pi and camera
module are tasked with capturing and transmitting a live video stream that can
be used for a variety of purposes, including safety mechanisms, visual
servoing, or simply entertainment. We have designed a custom protoboard to
house our electronic components. We have a variety of circuits on this board,
including power distribution, battery voltage monitoring, and control circuits
for the cannon solenoids and even a buzzer that acts as a horn.\\

The robot performs very well for its target application. The cannon has an
excellent range. We can launch t-shirts all the way across UTD's gym or we
scale back the range by decreasing the pressure in the tank. Our communication
range with the robot is limited by the mobile ad-hoc network we utilize. This
range could conceivably be increased by using existing network infrastructure,
such as the school's wireless network. However, the current network performance
is sufficient for our intended application. The robot is capable of moving
quite rapidly, so we actually have put mechanisms in place to limit its speed
due to safety and usability concerns.

\end{document}
