\documentclass[letterpaper,12pt]{article}

\usepackage[margin=1.0in]{geometry}
\usepackage{multirow}
\usepackage{tikz}
\usetikzlibrary{shapes.geometric, arrows, fit}

\newcommand{\specialcell}[2][c]{\begin{tabular}[#1]{@{}c@{}}#2\end{tabular}}

\begin{document}

\section{System Overview}

\begin{figure}[h!]
  \centering

  \tikzstyle{block} = [rectangle, rounded corners, minimum width=3cm, minimum height=2cm,text centered, draw=black]
  \tikzstyle{container} = [draw, rectangle, rounded corners, dashed, text width=11em, inner sep=3.5em]
  \tikzstyle{arrow} = [thick, ->, >=latex]
  \begin{tikzpicture}[node distance=2cm]
    \node (input) [align=center, xshift=-1cm] {User\\Inputs};
    \node (output) [right of=input, align=center] {Video\\Display};
    \node (computer) [block, yshift=-2cm] {Computer};
    \node (ar9331) [block, right of=computer, xshift=9cm, yshift=-2cm, align=center] {Atheros\\AR9331};
    \node (atmega) [block, below of=ar9331, yshift=-2cm, align=center] {Arduino\\ATmega32u4};
    \node (rpi) [block, right of=computer, above of=ar9331, xshift=-2cm, yshift=2cm] {Raspberry Pi};

    \node (yun) [container, fit=(atmega) (ar9331)] {};
    \node (yun_text) [yshift=-0.3cm] at (yun.north) {Yun};

    \draw [arrow] (computer.336) -- node[anchor=east, align=center, yshift=-5mm] {Velocity\\Commands} (ar9331.188);
    \draw [arrow] (ar9331.172) -- node[anchor=west, align=center, xshift=-12mm, yshift=5mm] {Status Updates} (computer.352);
    \draw [arrow] (ar9331.240) -- node[anchor=east, align=center] {GPIO\\Commands} (atmega.120);
    \draw [arrow] (atmega.60) -- node[anchor=west, align=center] {Sensor\\Readings} (ar9331.300);
    \draw [arrow] (rpi.172) -- node[anchor=east, align=center, yshift=5mm] {Video\\Stream} (computer.24);
    \draw [arrow] (computer.8) -- node[anchor=west, align=center, xshift=-12mm, yshift=-5mm] {Video Commands} (rpi.188);
    \draw [arrow] (input) -- (computer.134);
    \draw [arrow] (computer.46) -- (output);
  \end{tikzpicture}
  \caption{System Overview}
  \label{system_diagram}
\end{figure}

\section{Application Interface Specifications}

\subsection{Message Formats}

\begin{table}[h!]
  \centering
  \begin{tabular}{| c | c | c | c | c | c |}
    \hline
    \textbf{Field} & 0 & 1 & 2 & 3 & 4 \\
    \hline
    \textbf{Description} & \specialcell{Command\\Type} & \specialcell{Linear\\Velocity\\X} & \specialcell{Linear\\Velocity\\Y} & \specialcell{Linear\\Velocity\\Z} & \specialcell{Angular\\Velocity} \\
    \hline
    \textbf{Value} & 0 & -128 to 127 & -128 to 127 & -128 to 127 & -128 to 127 \\
    \hline
  \end{tabular}
  \caption{Velocity Command Message Format}
  \label{vel_cmd_msg}
\end{table}

\end{document}
