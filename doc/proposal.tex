\documentclass[letterpaper,12pt]{article}

\usepackage[margin=1in]{geometry}
\usepackage{tikz}
\usepackage{fancyhdr}
\usepackage{fixltx2e}
\pagestyle{fancy}
\fancyhf{}
\fancyhead[RE,LO]{EE 4389 Senior Design II Proposal}
\fancyhead[LE,RO]{Teleoperated Mobile Garment Accelerator}
\fancyfoot[R]{\thepage}
\fancyfoot[L]{Eckert, Hasan, Marcotte, and Rahman}

\setlength{\parindent}{0pt} % eliminate the need for \noindent everywhere

\usetikzlibrary{shapes.geometric, arrows, fit}

\begin{document}
\title{\textbf{Mobile Teleoperated Garment Accelerator}\\EE 4388 Senior Design II\\Proposal}
\author{Hazen Eckert \and Omar Hasan \and Ryan Marcotte \and Ridhwaan Rahman}
\date{2 February 2015}
\maketitle
\newpage

\section{Introduction}
\subsection{Brief Problem Description}
This project is focused on the design and implementation of a teleoperated, wheeled mobile robot that is capable of launching a t-shirt projectile. The robot will be holonomic and will consist of a rigid, rectangular chassis and a mounted cannon powered by $\textrm{CO}_2$. A human user will control all robot related tasks, including remotely driving, aiming, and launching the cannon. Both low-level motor controls and high-level coordination of motor speeds will be designed for controlled locomotion. Two on-board microcontrollers will be networked to the operator's computer, and video captured from a robot-mounted camera will be streamed to a heads-up display, which will allow for real-time control and monitoring of the robot. 

\subsection{Project Background and Application}
The Erik Jonsson School of Engineering often hosts events and sponsors collaborative projects amongst its students to promote engineering. Often, the school has used robotics based projects as a platform for these inspirational and educational events. Our project is an extension of these efforts as its product will be used by the robotics faculty to draw attention to the school of engineering at university events, which include basketball games, engineering week, freshman orientation, and the like.

\section{Problem Analysis}
The robot must wirelessly receive and perform velocity and t-shirt launching commands. To perform the commands, the robot must first have a mechanism by which to control its velocity and also a triggering mechanism for the t-shirt launcher. It must have a wireless receiver on-board and a camera for first person view streaming. A central controller should coordinate all robot tasks on-board and a power source for both the t-shirt launching mechanism as well as the components dedicated to maneuvering the robot. The following specifications will solve the problem as thus described:
\begin{itemize}
\item 4-wheel omnidirectional drive
\item Pneumatic t-shirt launcher with a shot range between 20 and 150 feet
\item Single power source for all electrical components
\item Single microcontroller board that includes a microcontroller and a microprocessor for embedded linux and wireless communication
\item Highly reconfigurable IP camera
\end{itemize}

\section{Objectives and Deliverables}
\subsection{Scope of Work}
\begin{itemize}
\item Design a robotic system capable of accomplishing tasks as described in the problem analysis.
\item Research, select, and obtain all components necessary for building the system.
\item Assemble all components and test to ensure proper functionality.
\item Develop software to enable user to interface with and control the system.
\end{itemize}

\subsection{Work Completed}
During Senior Design I, we accomplished the following:
\begin{itemize}
\item Researched necessary components to design an efficient system.
\item Selected and purchased required materials.
\item Assembled the mecanum chassis, including frame and gearboxes.
\item Designed and built pneumatics to allow for remote firing and controlled range.
\item Designed and assembled power distribution board prototype to power motors, microcontrollers, and pneumatics.
\item Demonstrated video streaming for remote viewing of the robot's point-of-view.
\end{itemize}

\subsection{Semester Goals}
During Senior Design II, we seek to accomplish the following:
\begin{itemize}
\item Drive the robot at a desired speed using speed controllers.
\item Fire t-shirts with electronic triggering.
\item Mount the t-shirt cannon on the robot chassis.
\item Design a user interface which allows for viewing the video stream from the robot, as well as issuing commands for movement and cannon firing.
\item Implement emergency kill switches to improve the safety of the robot.
\end{itemize}

\subsection{Deliverables}
By the end of this semester, we will produce the following deliverables:
\begin{itemize}
\item The assembled robot, with the following functionality:
  \begin{itemize}
  \item Capability to move laterally in response to remotely issued input commands.
  \item Fixed-position cannon capable of being triggered remotely.
  \item Emergency kill switch mechanism to manually disconnect power to the robot's motors.
  \end{itemize}
\item The accompanying software package, which will include:
  \begin{itemize}
  \item Graphical user interface with embedded video stream from robot and ability to issue remote commands.
  \item Network setup to allow for communication between user's computer and the robot.
  \item Utilities for user to monitor low-level robot activities.
  \end{itemize}
\end{itemize}

\section{Engineering Approach}
\subsection{} %add whatever section names you need to describe each aspect of the project. Sensors, Microcontrollers, Software are examples. 
\subsection{}
\subsection{}

\section{Team Roles}
\subsection{Roles}
Hazen Eckert - CE:
\begin{itemize}
\item Solenoid as triggering  mechanism
\item Electronic pressure sensing 
\item Testing safety capabilities of t-shirt cannon
\end{itemize}
Ryan Marcotte - EE:
\begin{itemize}
\item Software development on the Arduino Yun
\item Graphical User Interface implementation
\item Writing and sending weekly reports
\end{itemize}


Omar Hasan - EE:
\begin{itemize}
\item Chassis assembly
\item Power circuitry and electrical system
\item Driving the robot remotely using velocity commands
\item Speed controller testing
\end{itemize}
Ridhwaan Rahman - EE:
\begin{itemize}
\item Video streaming
\item Ethics statement
\item Speed controller testing
\end{itemize}
\subsection{Deadlines}
Our work deadlines are as follows:
\begin{enumerate}
\item February $9^{th}$ - Fire t-shirts using electronic triggering.
\item February $23^{th}$ - Drive the robot at a desired speed using speed controllers.
\item March $2^{nd}$ - Design a user interface which allows for viewing the video stream from the robot, as well as issuing commands for movement and cannon firing.
\item March $2^{nd}$ - Implement emergency kill switches to improve the safety of the robot.
\item April $27^{th}$ - Mount the t-shirt cannon on the robot chassis.
\end{enumerate}
\subsection{Budget}
Our total available budget is \$3,000. Thus far we have spent \$1898.4 on a microcontrollers,
chassis, t-shirt cannon, power distribution essentials, and pneumatics.
\section{Conclusion}
\end{document}
