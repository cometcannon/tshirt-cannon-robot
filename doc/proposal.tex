\documentclass[letterpaper,12pt]{article}

\usepackage[margin=1in]{geometry}
\usepackage{tikz}
\usepackage{fancyhdr}
\usepackage{fixltx2e}
\pagestyle{fancy}
\fancyhf{}
\fancyhead[RE,LO]{EE 4389 Senior Design II Proposal}
\fancyhead[LE,RO]{Teleoperated Mobile Garment Accelerator}
\fancyfoot[R]{\thepage}
\fancyfoot[L]{Eckert, Hasan, Marcotte, and Rahman}

\setlength{\parindent}{0pt} % eliminate the need for \noindent everywhere

\usetikzlibrary{shapes.geometric, arrows, fit}

\begin{document}
\title{\textbf{Mobile Teleoperated Garment Accelerator}\\EE 4388 Senior Design II\\Proposal}
\author{Hazen Eckert \and Omar Hasan \and Ryan Marcotte \and Ridhwaan Rahman}
\date{2 February 2015}
\maketitle
\newpage

\section{Introduction}
\subsection{Brief Problem Description}
This project is focused on the design and implementation of a teleoperated, wheeled mobile robot that is capable of launching a t-shirt projectile. The robot will be holonomic and will consist of a rigid, rectangular chassis and a mounted cannon powered by $\textrm{CO}_2$. A human user will control all robot related tasks, including remotely driving, aiming, and launching the cannon. Both low-level motor controls and high-level coordination of motor speeds will be designed for controlled locomotion. Two on-board microcontrollers will be networked to the operator's computer, and video captured from a robot-mounted camera will be streamed to a heads-up display, which will allow for real-time control and monitoring of the robot. 

\subsection{Project Background and Application}
The Erik Jonsson School of Engineering often hosts events and sponsors collaborative projects amongst its students to promote engineering. Often, the school has used robotics based projects as a platform for these inspirational and educational events. Our project is an extension of these efforts as its product will be used by the robotics faculty to draw attention to the school of engineering at university events, which include basketball games, engineering week, freshman orientation, and the like.

\section{Problem Analysis}
The robot must wirelessly receive and perform velocity and t-shirt launching commands. To perform the commands, the robot must first have a mechanism by which to control its velocity and also a triggering mechanism for the t-shirt launcher. It must have a wireless receiver on-board and a camera for first person view streaming. A central controller should coordinate all robot tasks on-board and a power source for both the t-shirt launching mechanism as well as the components dedicated to maneuvering the robot. The following specifications will solve the problem as thus described:
\begin{itemize}
\item 4-wheel omnidirectional drive
\item Pneumatic t-shirt launcher with a shot range between 20 and 150 feet
\item Single power source for all electrical components
\item Single microcontroller board that includes a microcontroller and a microprocessor for embedded linux and wireless communication
\item Highly reconfigurable IP camera
\end{itemize}

\section{Objectives and Deliverables}
\subsection{Scope of Work}
\begin{itemize}
\item Design a robotic system capable of accomplishing tasks as described in the problem analysis.
\item Research, select, and obtain all components necessary for building the system.
\item Assemble all components and test to ensure proper functionality.
\item Develop software to enable user to interface with and control the system.
\end{itemize}

\subsection{Work Completed}
During Senior Design I, we accomplished the following:
\begin{itemize}
\item Researched necessary components to design an efficient system.
\item Selected and purchased required materials.
\item Assembled the mecanum chassis, including frame and gearboxes.
\item Designed and built pneumatics to allow for remote firing and controlled range.
\item Designed and assembled power distribution board prototype to power motors, microcontrollers, and pneumatics.
\item Demonstrated video streaming for remote viewing of the robot's point-of-view.
\end{itemize}

\subsection{Semester Goals}
During Senior Design II, we seek to accomplish the following:
\begin{itemize}
\item Drive the robot at a desired speed using speed controllers.
\item Fire t-shirts with electronic triggering.
\item Mount the t-shirt cannon on the robot chassis.
\item Design a user interface which allows for viewing the video stream from the robot, as well as issuing commands for movement and cannon firing.
\item Implement emergency kill switches to improve the safety of the robot.
\end{itemize}

\subsection{Deliverables}
By the end of this semester, we will produce the following deliverables:
\begin{itemize}
\item The assembled robot, with the following functionality:
  \begin{itemize}
  \item Capability to move laterally in response to remotely issued input commands.
  \item Fixed-position cannon capable of being triggered remotely.
  \item Emergency kill switch mechanism to manually disconnect power to the robot's motors.
  \end{itemize}
\item The accompanying software package, which will include:
  \begin{itemize}
  \item Graphical user interface with embedded video stream from robot and ability to issue remote commands.
  \item Network setup to allow for communication between user's computer and the robot.
  \item Utilities for user to monitor low-level robot activities.
  \end{itemize}
\end{itemize}

\section{Engineering Approach}
\subsection{Chassis}
The robot has a 30" x 30" steel frame to which is attached four motor/wheel/gearbox assemblies. The motors and gearboxes we chose provide the robot with enough power to achieve a velocity of at least nine feet per second. We chose to use mecanum wheels to make our robot more maneuverable. The steel frame provides a sturdy mounting place for our cannon and our electronics.
\subsection{Cannon}
The cannon subsystem consists of the actual pneumatic cannon, the electronics triggering assembly, and the pressure regulation assembly. The pneumatic cannon is a commercial t-shirt cannon that triggers pneumatically and has a mechanically adjustable pressure. To trigger the cannon we replaced the mechanical pneumatic trigger with an electronic solenoid. To regulate the pressure the mechanical pressure regulator is set to the maximum pressure and that is supplied to the reservoir through an electronic solenoid. A pressure sensor monitors the pressure within the reservoir and once it reaches the desired pressure the solenoid will close, stopping the increase in pressure. This allows us to electronically set a desired pressure within the reservoir.
\subsection{Power}
To power our many components a 12VDC battery is used. A power distribution board provides the robot with up to twelve fuse protected circuits. The Raspberry Pi, the Arduino Yun and their associated sensors are powered by a 5V DC-to-DC buck converter that provides up to 5A. To power the solenoids a 24V DC-to-DC boost converter was utilized. Each of the 4 motors require their own speed controller. Our power distribution board provides 30A to each of these speed controllers. 
\subsection{Computation}
To manage all our sensors, motors, and the cannon we will use an Arduino Yun. This device was chosen because it has an integrated WiFi module. This allows us to more easily program the device and our control commands will be received over WiFi. This device will execute the commands given by the user as well as run a closed loop control over wheel velocities. This allows the user to set a desired speed and the controller will accelerate the wheel to match that velocity. We decided to do this so that we could regulate the speed for safety as well as to keep the controls consistent as the battery drained. \\

To manage the video stream we used a Raspberry Pi with its associated camera. We decided to use a Raspberry Pi over an wireless camera because it provides the opportunity for future engineers to implement higher level and more computationally expensive algorithms on board the robot. We chose it over other Linux based microprocessors because of its low cost and its large community support. 
\subsection{Sensors}
Two sensors are required for our robot. First, quadrature encoders provide information on wheel rotation to provide feedback for the velocity control algorithm. This sensor provides a signal when the wheel rotates a certain amount and indicates which direction. By aggregating these signals we can tell the number of rotations each wheel has made over time. Taking the first derivative of that value provides us with the angular velocity of each wheel. From this value we can derive an estimate of the robot's velocity.\\

 A pressure transducer provides our pressure regulation system the current pressure in the cannon's reservoir. When the pressure in the reservoir reaches the desired pressure set by the user, the flow of CO2 into the reservoir will be shut off. 
\subsection{Communication}
There are will be two communication pathways for this robot. First, the raspberry pi will be wirelessly transmitting the video stream over WiFi to the base station. This is accomplished through a password protected ad-hoc network. This solution was chosen because it requires less hardware than a traditional network as a router is not necessary. 
Second, the Arduino Yun will receive control commands from and send status/diagnostic information to the base station. This will also be implemented as an ad-hoc network. The linux-based Atheros chip on the Yun will be responsible for relaying the network commands to the ATmega over serial.
\subsection{Control}
The robot is controlled through a graphical user interface and a hand-held gaming controller. The graphical user interface presents the user with the video feed, speed of the robot, cannon status, and any diagnostic/warning messages.  The game pad's joysticks will be used to control the velocity and rotation of the chassis. The trigger buttons will fire the pneumatic cannon.
\subsection{Safety Requirements}
The robot and cannon must be safe to operate at a public event. There will be software and hardware safety components. The robot will automatically stop when the connection with the base station is lost. There will be a software defined limit to the speed and acceleration of the wheels. An emergency power switch will be easily accessible and cut off all power to the motors when activated. The pneumatic cannon will have electronic pressure sensing and if the pressure withing the reservoir is out of a desired range the air supply will be cut off and a warning will be displayed to the user. A pressure release valve will prevent the pressure from exceeding the rating of any components.

\section{Team Roles}
\subsection{Roles}
Hazen Eckert - CE:
\begin{itemize}
\item Solenoid as triggering  mechanism
\item Electronic pressure sensing 
\item Testing safety capabilities of t-shirt cannon
\end{itemize}
Ryan Marcotte - EE:
\begin{itemize}
\item Software development on the Arduino Yun
\item Graphical User Interface implementation
\item Writing and sending weekly reports
\end{itemize}


Omar Hasan - EE:
\begin{itemize}
\item Chassis assembly
\item Power circuitry and electrical system
\item Driving the robot remotely using velocity commands
\item Speed controller testing
\end{itemize}
Ridhwaan Rahman - EE:
\begin{itemize}
\item Video streaming
\item Ethics statement
\item Speed controller testing
\end{itemize}
\subsection{Deadlines}
Our work deadlines are as follows:
\begin{enumerate}
\item February $9^{th}$ - Fire t-shirts using electronic triggering.
\item February $23^{th}$ - Drive the robot at a desired speed using speed controllers.
\item March $2^{nd}$ - Design a user interface which allows for viewing the video stream from the robot, as well as issuing commands for movement and cannon firing.
\item March $2^{nd}$ - Implement emergency kill switches to improve the safety of the robot.
\item April $27^{th}$ - Mount the t-shirt cannon on the robot chassis.
\end{enumerate}
\subsection{Budget}
Our total available budget is \$3,000. Thus far we have spent \$2100 on a microcontrollers,
chassis, t-shirt cannon, power distribution essentials, and pneumatics.
\section{Conclusion}
Last semester we focused on the design of our system such that this design, besides fulfilling its requirements, allowed for modularity. This will allow us to expand on the design's minimum requirements of remote control and t-shirt launching, so that we can explore interesting engineering problems, such as visual servoing and reloading mechanisms. The quick acquisition of the robot's materials gave us opportunities to test our system in advance of more complicated problems we will face.\\

This semester we plan on the implementation and expansion of our design. We expect to have most robot functionality implemented by the beginning of April to allow us a month of troubleshooting, documentation, and testing for robustness. 

\end{document}
